%%This is a very basic article template.
%%There is just one section and two subsections.
\documentclass[serif,mathserif]{article}
\usepackage[utf8x]{inputenc}
\usepackage{comment}
\usepackage[portuges]{babel}
\usepackage{listings}
\usepackage{dcolumn}
\lstloadlanguages{Ruby}
\lstdefinelanguage{Smalltalk}{ 
  morekeywords={true,false,self,super,nil}, 
  sensitive=true, 
  morecomment=[s]{"}{"}, 
  morestring=[d]', 
  style=SmalltalkStyle 
} 
\lstdefinestyle{SmalltalkStyle}{ 
  literate={:=}{{$\gets$}}1{^}{{$\uparrow$}}1 
}

% template

\author{
    \\ Rodrigo di Lorenzo Lopes \\  \texttt{rodrigo.lorenzo@abril.com.br}
	\and 
    \\ Eduardo Nicola F. Zagari \\ \texttt{zagari@abril.com.br}
}
\title{Curso: Ruby Básico}

\begin{document}
\maketitle
 
\tableofcontents

\section{Introducao: Ruby.new}
\begin{frame} 

\begin{itemize}
\item Linguagem para humanos
\item Compare:
\begin{lstlisting}[language=ruby]
5.times { print "Ola!" }
\end{lstlisting}

\begin{lstlisting}[language=java]
for (int i=0; i <10; i++) { printf("Ola!");}
\end{lstlisting}

\end{itemize}

\end{frame}

\subsection{\#\#Estruturas básicas}
\begin{frame} 

\begin{itemize}
  \item Variáveis 
\begin{lstlisting}[language=ruby]
x, y, taxa_do_lixo2
\end{lstlisting}
   \item{Numeros}
\begin{lstlisting}[language=ruby]
1, -1.2, 6.03e-23    
\end{lstlisting}
	\item {String}
\begin{lstlisting}[language=ruby]
"alguma coisa assim"
%q(veremos outras formas de declarar strings)
\end{lstlisting}
 
\end{itemize}

\end{frame}

\subsection{\#\#Estruturas basicas}
\begin{itemize}
  \item Symbols
% * Symbols
\begin{lstlisting}[language=ruby]
:x, :y, :isso_parece_uma_string
\end{lstlisting}

  \item Constantes 
\begin{lstlisting}[language=ruby]  
EmpireStateBuilding, NEA, PI
\end{lstlisting}

  \item Objetos especiais 
\begin{lstlisting}[language=ruby]
true, false, nil
\end{lstlisting}
\end{itemize}

Símbolos são alocados uma única vez: :a.object\_id durante uma execução sempre
retornara o mesmo valor. Isso nao acontece com string. O método equal? so
devolve true se dois objetos são de fato o mesmo objeto (e instâncias da mesma
classe com valores iguais).

\subsection{\#\#Metodos (mensagens)}
\begin{lstlisting}[language=ruby]
i = 1        
texto = "um texto"; puts texto
a = b = c = 0
1 == 2           # sugar syntax!!!
# metodo de classe
1.methods  # lista todos os metodos daquele objeto
1.send(:even?) # outra forma de enviar mensagens
def fibo(n = 1)
    fibo(n-2) + fibo(n-1) if n >= 2
end
def self.log
  puts "metodo de classe"
end
\end{lstlisting}

Lembre-se ... voce pode redefinir um método
Quase tudo e objeto

\subsection{\#\#Estruturas de Controle - if}
 
       
\subsubsection {Exemplo Completo}
\begin{lstlisting}[language=ruby]
if count > 10
  puts "Try again"
elsif tries == 3
  puts "You lose"
  puts  Number:"
end
\end{lstlisting}

\subsubsection {Exemplo Simples}
\begin{lstlisting}[language=ruby]
if radiation > 3000
  puts "Danger"
end
\end{lstlisting}

\subsubsection {Modificador de Sentenca}
\begin{lstlisting}[language=ruby]
puts "Danger, Will Robinson" if radiation > 3000
\end{lstlisting}

\subsection{\#\#Estruturas de Controle - case} 
\begin{lstlisting}[language=ruby]
print "Enter your grade: "
grade = gets.chomp
case grade
when "A"
  puts 'Well done!'
when "B"
  puts 'Try harder!'
when "C", "D"
  puts 'You need help!!!'
  puts "You just making it up!"
 end
\end{lstlisting}


\subsection{\#\#Estruturas de Controle - while}
\begin{lstlisting}[language=ruby]
while weight < 100 and numPallets <= 30
  pallet = nextPallet()
  weight += pallet.weight
  numPallets += 1
end
\end{lstlisting}
 
\subsubsection {Modificador de Sentenca}
\begin{lstlisting}[language=ruby]
square = square*square  while square < 1000
\end{lstlisting}

\subsection{\#\#Estrutura de Controle -  for}
\begin{lstlisting}[language=ruby]
for i in 0..5
   puts "Value is #{i}"
end
\end{lstlisting}
\subsection{\#\#Estrutura de Controle -  until} 
\begin{lstlisting}[language=ruby]
until weight >= 100 ||numPallets > 30
  pallet = nextPallet()
  weight += pallet.weight
  numPallets += 1
end
\end{lstlisting}

\subsubsection {Modificador de Sentenca}
\begin{lstlisting}[language=ruby]
square = square*square  until square >= 1000
\end{lstlisting}

\subsection{\#\#Containers - Array}
\begin{lstlisting}[language=ruby]
a = [ 3.14159, "pie", 99 ]
a.type               #        Array
a.length               #        3
a[0]                  #        3.14159
a << 1
a[3]                  #        1
a[-2]                 #        99
b = Array.new
b << a               #   [[3.14159, "pie", 99, 1]]
b[0..3] = a        #   [3.14159, "pie", 99, 1]
b[0, 2] = 1        #   [1, 1]
c = %w{a b c d e }                #  =>  ["a", "b", "c", "d"]
\end{lstlisting}


\subsection{\#\#Containers - Hash}
\begin{lstlisting}[language=ruby]
h = {'dog' => 'canine', 'cat' => 'feline', 'donkey' => 'asinine'}
h.length        #        3
h['dog']        #        "canine"
h['cow'] = 'bovine'
h[12]    = 'dodecine'
h['cat'] = 99
h        # => {"cow"=>"bovine", "cat"=>99, 12=>"dodecine",
"donkey"=>"asinine", "dog"=>"canine"}
  
a = [[1, 'a'],[2, 'b'],[3, 'c'], [4, 'd']]
b = Hash[a]
# => {1=>"a", 2=>"b", 3=>"c", 4=>"d"}
\end{lstlisting}


\subsection{\#\#Blocos e Iteradores}

Passando blocos
\begin{lstlisting}[language=ruby]
(1..12).each { |i| puts i}
[1, 2, 4].each do |i|
    puts i
end
\end{lstlisting}

Blocos de codigo
\begin{lstlisting}[language=ruby]
(1..20).each{|x| puts x}
\end{lstlisting}

Influencia do Smalltalk:
\begin{lstlisting}[language=smalltalk]
   1 to: 20 do: [:x | x printN1]
\end{lstlisting}

\subsection{\#\# Métodos de um Enumerable}
\begin{lstlisting}  
all?, any?, collect, detect, each_cons, each_slice, each_with_index, entries,
enum_cons, enum_slice, enum_with_index, find, find_all, grep, include?, inject,
map, max, member?, min, partition, reject, select, sort, sort_by, to_a,
to_set, zip
\end{lstlisting}
 
\subsection{\#\#Exemplos com Enumeraveis}
\begin{lstlisting}[language=ruby]
names = %w{ Frye Leela Zoidberg }
names.find {|name|  name.length>4}          # => "Leela"
names.find_all { |name| name.length > 4}
     #=> ["Leela", "Zoidberg"]
names.grep /oidberg/
# => ["Zoidberg"]
names.group_by {|name|  name.length}
   # =>  {4=>["Frye"], 5=>["Leela"], 8=>["Zoidberg"]}
\end{lstlisting}

\subsection{\#\#Mais exemplos com Enumeraveis}
\begin{lstlisting}[language=ruby]
names = %w{ Frye Leela Zoidberg }
names.map {|name| name.downcase}
# => ["frye", "leela", "zoidberg"]
names.reduce {|acc, name| name.length <= 5 ? acc + name : acc }
# => "FryeLeela"
names.join ", "
# => "Frye, Leela, Zoidberg"
\end{lstlisting}


\subsection{\#\#Invocando blocos}
\begin{lstlisting}[language=ruby]
def proxy_method
  puts "Calling command at: #{Time.new}"
  yield
proxy_method { puts "hello world proxified! "}
#ou com paremtros
def proxy_method
   yield(Time.new)
proxy_method {|time| puts "hello world proxified  at #{time}"}
\end{lstlisting}


\subsection{\#\#Invocando blocos II}
\begin{lstlisting}[language=ruby]
def proxy_method(&method)
  # argumento com & precisa ser o ultimo da lista
  puts "Calling command at: #{Time.new}"
  method.call
proxy_method { puts "hello world proxified! "}
#ou com paremtros
def proxy_method (&method)
   method.call(Time.new)
proxy_method {|time| puts "hello world proxified  at #{time}"}
\end{lstlisting}

\subsection{\#\#Proc x Lambda}
\begin{lstlisting}[language=ruby]
fx = Proc.new {|x| x**2}
fxy = proc {|x,y| x+y}
# calling
fx.call(2) # => 4
fxy[2,3,4] #=> 5
fx = lambda {|x| x**2}
fxy = lambda {|x,y| x+y}
# calling
fx.call(2) # => 4
fxy.call(2,3,4) #=> exception na cara!
Proc.new e proc sao equivalentes
\end{lstlisting}


\subsection{\#\#Lambda ``Calculus''}

\begin{lstlisting}[language=ruby, caption="Derivada em Ruby"]
def d(f)
   lambda {|a|
     h = 0.0000000001 # um valor pequeno para h
     h = h * a       if a < 1 && 0 < a
     (f[a+h]-f[a])/h
   }
f = lambda {|x| x**2}
puts d(f)[4]
\end{lstlisting}

\subsection{\#\#Objetos em Ruby}
\begin{lstlisting}[language=ruby]
class BookInStock
  def initialize(isbn, price)
    @isbn = isbn
    @price = Float(price)
  end
 
  def to_s
    "ISBN:#{@isbn}, price: #{@price}"
  end 
stock = BookInStock.new
# ou
stock = BookInStock.new (1234, 10.39)
#invocando metodo
puts stock.to_s
\end{lstlisting}


\subsection{\#\#Variaveis e Escopo}

\begin{tabular}{ l | c | r }
Variáveis Locais & \verb|x name thx1138 _x _26| \\
Variáveis de Instancia & \verb|@name @X  @_ @plan9| \\
Variáveis de Classe & \verb|@@total @@N @@x_pos| \\
Variáveis Globais & \verb|$debug $CUSTOM $_ $plan9| \\
Nomes de Classe & \verb|String BigDecimal| \\
Constants & \verb|FEET_PER_MILE DEBUG|
\end{tabular}

\subsection{\#\#Atributos de instância}

\subsubsection{forma tradicional}
\begin{lstlisting}[language=ruby]
class BookInStock  
  def isbn
    @isbn
  end
 
  def isbn=(value)
    @isbn = value
  end
 
  def price
    @price
  end
\end{lstlisting}

\subsubsection{forma declarativa}
\begin{lstlisting}[language=ruby]
class BookInStock  
  attr_accessor :isbn
  attr_reader :price
 end

\end{lstlisting}

\subsection{\#\#Herança}
\subsubsection{Exemplo de Heranca}
\begin{lstlisting}[language=ruby]
class SpecialStock < BookInStock
\end{lstlisting}

\subsection{\#\#Herança - Singleton Pattern}

\subsubsection{forma tradicional}
\begin{lstlisting}[language=ruby]
class Logger
  private_class_method :new
  @@logger = nil
  def Logger.create
    @@logger = new unless @@logger
    @@logger
  end
end
\end{lstlisting}

\subsubsection{módulo Singleton}
\begin{lstlisting}[language=ruby]
require 'singleton'
class Logger
  include Singleton
  
  def initialize
    @log = File.open("log.txt", "a")
  end
  def log(msg)
    @log.puts(msg)
  end
Logger.instance.log('message 2')
\end{lstlisting}

\begin{lstlisting}[language=ruby]
stock =  BookInStock.new
class << stock
   def alter_price
         price * 1.4
   end
\end{lstlisting}

\subsection{\#\#Criando um Enumerable (I)}
  *  Basta implementar o metodo each. 
\begin{lstlisting}[language=ruby]
class Node
  include Enumerable 
  attr_accessor :next, :previous, :v
 def initialize(v = {})
    @v = v
  end
 def to_s
    v.to_s
  end
\end{lstlisting}

\subsection{\#\#Criando um Enumerable (II)}
\begin{lstlisting}[language=ruby]
linked_list.rb (continuacao)
 def <<(node)
    node.next = self.next
    node.previous = self
    self.next.previous = node unless self.next.nil?
    self.next = node
  end
 def remove
    node = self.previous
    node.next = self.next
    self.next.previous = node
    self
 end
\end{lstlisting}

\subsection{\#\#Criando um Enumerable (III)}
\begin{lstlisting}[language=ruby]
  def each
    node = self.next
    until node == self || node.nil?
      yield node
      node = node.next
    end
  end

\end{lstlisting}

\subsection{\#\#Criando um Enumerable (II)}

% linked_list.rb (continuacao)

\begin{lstlisting}[language=ruby]
def <<(node)
 node.next = self.next
 node.previous = self
 self.next.previous = node unless self.next.nil?
 self.next = node
end
def remove
 node = self.previous
 node.next = self.next
 self.next.previous = node
 self
end
\end{lstlisting}

\end{document}
