\documentclass[serif,mathserif]{book}
\usepackage[utf8]{inputenc}
\usepackage{comment}
\usepackage[portuges]{babel}
\usepackage{booktabs}
\usepackage{listings}
\usepackage{listingsutf8}
\usepackage{xcolor}
\usepackage{bashful}
\usepackage{dcolumn}
\usepackage{hyperref}
\usepackage[stable]{footmisc}

\usepackage{caption}
\DeclareCaptionFont{white}{\color{white}}
\DeclareCaptionFormat{listing}{\colorbox{gray}{\parbox{\textwidth}{#1#2#3}}}
\captionsetup[lstlisting]{format=listing,labelfont=white,textfont=white}


\lstset{%
        frame=tb,
    	extendedchars=true,
    	inputencoding=utf8/latin1,
        literate=%
        {é}{{\'{e}}}1
        {è}{{\`{e}}}1
        {ê}{{\^{e}}}1
        {ë}{{\¨{e}}}1
        {É}{{\'{E}}}1
        {Ê}{{\^{E}}}1
        {û}{{\^{u}}}1
        {ú}{{\'{u}}}1        
        {ù}{{\`{u}}}1
        {â}{{\^{a}}}1
        {à}{{\`{a}}}1 
        {á}{{\'{a}}}1
        {ã}{{\~{a}}}1
        {Á}{{\'{A}}}1
        {Â}{{\^{A}}}1
        {Ã}{{\~{A}}}1
        {ç}{{\c{c}}}1
        {Ç}{{\c{C}}}1
        {õ}{{\~{o}}}1
        {ó}{{\'{o}}}1
        {ô}{{\^{o}}}1
        {Õ}{{\~{O}}}1
        {Ó}{{\'{O}}}1
        {Ô}{{\^{O}}}1
        {î}{{\^{i}}}1
        {Î}{{\^{I}}}1
        {í}{{\'{i}}}1
        {Í}{{\~{Í}}}1
}

\lstloadlanguages{Ruby}
\lstdefinelanguage{Smalltalk}{ 
  morekeywords={true,false,self,super,nil}, 
  sensitive=true, 
  morecomment=[s]{"}{"}, 
  morestring=[d]', 
  style=SmalltalkStyle 
} 
\lstdefinestyle{BashInputStyle}{
  language=bash,
  numbers=none,  
  basicstyle=\small\sffamily,
  frame=tb,
  columns=fullflexible,
  backgroundcolor=\color{yellow!20},
  % linewidth=0.9\linewidth,
  xleftmargin=0.1\linewidth
}

\lstdefinestyle{BashOutputStyle}{
  basicstyle=\small\ttfamily,
  numbers=none,
  frame=tblr,
  columns=fullflexible,
  backgroundcolor=\color{blue!10},
  % linewidth=0.9\linewidth,
  xleftmargin=0.1\linewidth
}

\lstdefinestyle{SmalltalkStyle}{ 
  literate={:=}{{$\gets$}}1{^}{{$\uparrow$}}1 
}

\newcommand*{\Package}[1]{\texttt{#1}}%
\renewcommand*\lstlistingname{Listagem}

\author{ 
    \\ Rodrigo di Lorenzo Lopes \\  \texttt{rodrigo.lorenzo@abril.com.br}
	\and 
    \\ Celestino Ferreira Gomes \\ \texttt{contato@tinogomes.com}
}
\title{Curso: Ruby Básico}

\begin{document}
\renewcommand*{\toclevel@chapter}{-1}

\maketitle
 
\tableofcontents

\chapter{Introducao: Ruby.new}

\section{Ideia do Curso}
Fizemos esse curso com foco no profissional que já sabe programar em alguma linguagem e compreende
os princípios de desenvolvimento de algoritmos. Para esses profissionais, cursos básicos de linguagens de computação
são sempre entendiantes e lentos. Emprega-se muito tempo com coisas triviais enquanto que coisas mais interessantes
são deixadas de lado. O profissional acaba tendo que aprender o que realmente importa por conta própria.

É fato que o aprendizado é um fenômeno pessoal, mas acreditamos que nossa função é expor ao aluno os 
detalhes da linguagem que a torna interessante. Esse curso tem essa ambição e iremos propor desafios 
que consideramos interessantes para dinâmicas como Dojo e para quem queira resolve-los sozinho.

Em cada capítulo, tentaremos apresentar o conceito da linguagem e propor problemas ou desafios em que esses conceitos
possam  ser aplicados. 

\section{Sobre Ruby}
Ruby é uma linguagem de programação \em{interpretada}, de \em{tipagem dinâmica} e \em{forte}, com \em{gerenciamento de memória automático},
 originalmente planejada e desenvolvida no Japão em 1995, por Yukihiro "Matz" Matsumoto, para ser usada como linguagem de script.

Matz queria uma linguagem de script que fosse mais poderosa do que Perl, e mais orientada a objetos do que Python.
Ruby é primariamente, uma linguagem \em{orientada a objetos}, mas suporta outros paradigmas de programação,
como \em{funcional}, \em{imperativa}  e \em{reflexiva}.

Foi inspirada principalmente por Python, Perl, Smalltalk, Eiffel, Ada e Lisp, sendo muito similar em vários aspectos a Python.

fonte: \href{http://en.wikipedia.org/wiki/Ruby_(programming_language)}{Wikipedia - http://en.wikipedia.org/wiki/Ruby_(programming_language)}

\section{Inslatação}
Antes de prosseguir com o curso, vejamos como instalar o interpretador da linguagem. Iremos utilizar o MRI, o interpretador
Ruby do Matz mas o leitor pode se aventurar por outros interpretadores. O MRI é hoje portado para os 
principais sistemas operacionais e de fácil instalação.

\subsection{Windows}

Baixar o executável de instalação em \url{http://rubyinstaller.org/downloads/}

\subsection{Linux (Debian / Ubuntu)}

\begin{lstlisting}[style=BashInputStyle]
$ [sudo] apt-get install ruby1.9.1
\end{lstlisting}

\subsection{Mac (via Homebrew)\footnote{Linux e Mac normalmente já vem com uma instalação de Ruby.}}

\begin{lstlisting}[style=BashInputStyle]
$ brew install ruby
\end{lstlisting}

Podemos verificar a versão já instalada do Ruby que está no \Package{PATH}.

Verificar a versão de ruby instalada, digite  \Package{ruby --version}.

\begin{lstlisting}[style=BashInputStyle]
$ ruby --version
\end{lstlisting}

Você receberá uma mensagem como essa:
\begin{lstlisting}[style=BashOutputStyle]
ruby 1.9.3p374 (2013-01-15 revision 38858) [x86_64-darwin10.8.0]
\end{lstlisting}

\Package{ruby -v} também imprime a versão do ruby, porém ativa o modo \em{verbose} e permite executar scripts.

O código-fonte ruby é um arquivo de texto como qualquer outro. Você pode utilizar seu editor de texto favorito, mas aqui a lista dos principais editores e IDEs
disponíveis para a linguagem: 

\begin{itemize}
\item \href{http://www.sublimetext.com/2}{Sublime Text 2}
\item \href{http://www.vim.org/}{Vim}
\item \href{http://www.gnu.org/software/emacs/}{Emacs}
\item \href{https://github.com/textmate/textmate/}{TextMate 2 (Mac)}
\item \href{https://netbeans.org/downloads/6.9.1/index.html}{Netbeans 6.9.1}
\item \href{http://www.aptana.com/}{Aptana}
\end{itemize}

 

\section{Ruby.new}

Ao longo desse curso, veremos comos os paradigmas de programação são aplicados em Ruby. Por enquanto, para exemplificar o pardigma orienta-a-objetos,
observe o exemplo:

Laço em Ruby
\begin{lstlisting}[language=ruby]
5.times { print "Ola!" }
\end{lstlisting}

Laço em Java
\begin{lstlisting}[language=java]
for (int i=0; i <10; i++) { printf("Ola!");}
\end{lstlisting}

A consistência de ``tudo ser objeto''\cite{abadi_theory_1996} parte dos príncipios de legibilidade e expressividade 
(acima por exemplo da eficiência em tempo de execução). É quase possível entender o código sem mesmo conhecer a linguagem. 
Veja outro exemplo:

\begin{lstlisting}[language=java]
   exit unless "restaurante".include? "aura"
   ['toasty', 'cheese', 'wine'].each 
        { |food| print food.capitalize }
\end{lstlisting}

\section{Sobre a sintaxe}

Algumas itens sobre a sintaxe são apenas conveções. Embora, pode-se utilizar um identificador arbitrário para classes 
e métodos, é importante manter a consistência com os programas da linguagem.
 
Um identificador em Ruby é uma sequência de caracteres que segue o padrão \verbatim{[a-zA-Z_][a-zA-Z0-9_]*}.
 
\begin{itemize}
  \item  Espaços, tabulações e blocos de comentários serão ignorados pelo interpretador.
  \item  Ponto-e-vírgula e nova linha é considerado um novo comando. Porém, se for encontrado um operador, como `+', `-', ou o sinal de `\textbackslash', indica que a 
próxima linha é continuação do comando atual.
  \item  Os identificadores são nomes de variáveis, constantes e métodos e é \em{case-sensitive}, ou seja, `DNS' e `Dns` são duas constantes diferentes.
  \item  Deve-se usar letras\_minúsculas para nomes de variáveis e métodos. `nome\_completo'
  \item  Deve-se usar LETRAS\_MAIÚSCULAS para nome de CONSTANTES. `MAX\_ATTENDEES = 100'
  \item  Deve-se usar \em{CamelCase} para nome de Classes. `NotificationMailer'
\end{itemize}

\section {Running Ruby}

Vamos criar uma arquivo Hello World! do Ruby.

\begin{lstlisting}[style=BashInputStyle]
$ echo 'puts "hello world"' > hello_word.rb
$ ruby hello_word.rb
\end{lstlisting}

A saída será: 

\begin{lstlisting}[style=BashOutputStyle]
hello world
\end{lstlisting}

Como ocorre com interpretadores como Perl, é possível passar o script por argumento na linha de comando:

\begin{lstlisting}[style=BashInputStyle]
$ ruby -e 'puts "hello world"'
\end{lstlisting}

A saída será a mesma:

\begin{lstlisting}[style=BashOutputStyle]
hello world
\end{lstlisting}

O ruby tem ainda um shell interativo (IRB) que pode ser usado para realizar testes:


\begin{lstlisting}[style=BashInputStyle]
$ irb
>> puts 'hello world'
hello world
=> nil
>>
\end{lstlisting}

\chapter{Estruturas básicas}

Nessa sessão veremos as estruturas básicas da linguagem: declarações, comentários, estruturas de controle e literais. 

Em Ruby, tudo que pode ser atribuído a uma variável é um objeto e todo objeto tem uma classe. Então todas as literais 
(números, expressões regulares, strings) também são objetos.

É bom lembrar que numa linguagem orientada a objetos, a forma básica de interagir com 
objetos é por meio de mensagens. Cada objeto ``entende'' um certo conjunto de mensagens definidos em métodos.
 A executação de um método se dá no envio da mensagem 
com o nome do método para o objeto. Em particular, toda classe ``entende" todos os métodos da classe Object.
 Veremos tudo isso em detalhes. Mas tomemos agora 
o método class e object. Esse método devolve a classe que o objeto pertence (ou qual classe foi utilizada para instanciar o objeto). Veja:
 

\begin{lstlisting}[language=ruby]
>> 1.class
=> Fixnum
>> "texto".class
=> String
>> :simbolo.class
=> Symbol
>> true.class
=> TrueClass
>> false.class
=> FalseClass
>> nil.class
=> NilClass
>> Object.class
=> Class
>> Class.class
=> Class
\end{lstlisting}


Já falamos que em Ruby, TUDO é objeto?

\begin{lstlisting}[language=ruby]
>> 1.methods.count
=> 130
> 1.methods.sort
=> [:!, :!=, :!~, :%, :&, :*, :**, :+, :+@, :-, :-@, :/, :<, 
:<<, :<=, :<=>, :==, :===, :=~, :>, :>=, :>>, :[], :^, 
:__id__, :__send__, :abs, :abs2, :angle, :arg, :between?, 
:ceil, :chr, :class, :clone, :coerce, :conj, :conjugate, ...]
>> 'texto'.methods.count
=> 162
>> 'texto'.methods.sort
=> [:!, :!=, :!~, :%, :*, :+, :<, :<<, :<=, :<=>, :==, :===,
:=~, :>, :>=, :[], :[]=, :__id__, :__send__, :ascii_only?, 
:between?, :bytes, :bytesize, :byteslice, ...]
\end{lstlisting}

\section{Comentários}

A forma mais comum de comentário em Ruby é o comentário de uma linha. Os comentários encontrados no 
código-fonte do interpretador Ruby do Matz (MRI, do inglês Matz's Ruby Interpreter) são da forma de   
múltiplos comentários de uma linha ao invés de comentários de múltiplas linhas.

Como guia de estilo recomendamos que você utilize a primeira forma (comentários de uma linha), porque
é visualmente mais fácil identificá-los, mas caso queira ser do contra, utilize a segunda forma.

\subsection{Uma linha}
\begin{lstlisting}[language=ruby]
 # Esta linha é um comentário.
 1 + 1 # este texto a direita do sinal de tambem é um comentário.
\end{lstlisting}

\subsection{Múltiplas linhas}

\begin{lstlisting}[language=ruby]
=begin
O texto envolvido por =begin e =end é comentário.
Mas para isto funcionar, o =begin e =end devem estar
na exterma esquerda do seu código, ou seja, na coluna
0 (zero).
=end
\end{lstlisting}

\section{Números}

\subsection{Inteiros}

\begin{lstlisting}[language=ruby]
123                       # Inteiro (Fixnum)
-123                      # Inteiro negatico (Fixnum)
1_123                     # Inteiro (Fixnum)
123_456_789_123_456_789   # Inteiro (Bignum)
0xAB                      # Número Hexadecimal (170)
0377                      # Número Octal (255)
0b001001                  # Número binário (9)
\end{lstlisting}

\subsection{Pontos Flutuantes}

\begin{lstlisting}[language=ruby]
123.45                    # Número com ponto flutuante (Float)
1.2e-3                    # Número com ponto flutuante (0.0012)
\end{lstlisting}

\section{String} 

\subsection{String 'single quotes'}

\begin{lstlisting}[language=ruby] 
>> puts 'texto'
texto
>> puts 'texto'.length
5
>> puts 'texto'.upcase
TEXTO
>> puts 'tex'.+('to')
texto
>> puts 'tex' + 'to' # syntax sugar
texto
>> puts 'tex' << 'to'
texto
>> String.new << 'texto'
texto
>> 'tex%s' % 'to'
texto
\end{lstlisting}

Para usar os caracteres '\'' ou '\textbackslash', você pode usar sequência de \em{escape}  '\textbackslash\'' e '\textbackslash\textbackslash'.

\begin{lstlisting}[language=ruby]
>> puts 'texto \' \\' 
texto ' \
\end{lstlisting}

\subsection{String 'double quotes'}

Existe uma diferença entre construir strings com aspas simples e aspas duplas. Strings montadas com aspas duplas, aceitam interpolação de conteúdo para construir a string final.

\begin{lstlisting}[language=ruby]
>> puts "o resultado de 1 + 1 é #{ 1 + 1 }."
o resultado de 1 + 1 é 2.
>> puts 'o resultado de 1 + 1 é #{ 1 + 1 }.'
o resultado de 1 + 1 é #{ 1 + 1 }.
\end{lstlisting}

O valor da expressão a ser interpolada, será o resultado do método 'to\_s' do objeto.

\subsection{Sequências de \em{escape}}

\begin{lstlisting}[language=ruby]
>> puts "      world\rhello"
hello world
>> puts "\thello \b\sworld"
    hello world
>> puts '\thello \b\sworld'
\thello \b\sworld
\end{lstlisting}

\begin{itemize}
  \item  '\textbackslash"' – double quote
  \item  '\textbackslash\textbackslash' – single backslash
  \item  '\textbackslash{}a' – bell/alert
  \item  '\textbackslash{}b' – backspace
  \item  '\textbackslash{}r' – carriage return
  \item  '\textbackslash{}n' – newline
  \item  '\textbackslash{}s' – space
  \item  '\textbackslash{}t' – tab
\end{itemize}


\subsection{String multiplas linhas}

\begin{lstlisting}[language=ruby]
>>> puts <DOC
 Esta é uma string em múltiplas linhas.
     * item
     * item
     * item
 DOC
\end{lstlisting}

Resultado:
\begin{lstlisting}[style=BashOutputStyle]
 Esta é uma string em múltiplas linhas.
     * item
     * item
     * item
\end{lstlisting}

Se quiser identar o finalizaror, para usar '<<-'.

\begin{lstlisting}[style=BashOutputStyle]
>>> puts <<-DOC
 Esta é uma string em múltiplas linhas.
     * item
     * item
     * item
     DOC
\end{lstlisting}

Resultado:
 \begin{lstlisting}[style=BashOutputStyle]
 Esta é uma string em múltiplas linhas.
     * item
     * item
     * item 
 \end{lstlisting}

\section{Símbolos}

Os símbolos são ideais para serem usados como chave em 'Hash'.

\begin{lstlisting}[language=ruby]
:x, :y, :chave
\end{lstlisting}

Símbolos são alocados uma única vez: ':a.object\_id' durante uma execução sempre retornara o mesmo valor.

Isso não acontece com string.

O método 'equal?' só devolve 'true' se dois objetos são de fato o mesmo objeto (e instâncias da mesma classe com valores iguais).

\begin{lstlisting}[language=ruby]
1.equals?(1)             # => true
:key.equals?(:key)       # => true
"texto".equals?("texto") # => false
\end{lstlisting}

\section{Statement}
Ruby é uma linguagem imperativa. Todas as declarações são nada mais que comandos. Todos os comandos devolvem um
valor. O resultado de uma atribuição por exemplo é o valor da atribuição. O Resultado de uma ``declaração'' de classe
é o objeto da classe. 

\section{Métodos (mensagens)}
\begin{lstlisting}[language=ruby]
i = 1        
texto = "um texto"; puts texto
a = b = c = 0
1 == 2           # sugar syntax!!!
# metodo de classe
1.methods  # lista todos os metodos daquele objeto
1.send(:even?) # outra forma de enviar mensagens
def fibo(n = 1)
    fibo(n-2) + fibo(n-1) if n >= 2
end
def self.log
  puts "metodo de classe"
end
\end{lstlisting}

Lembre-se ... voce pode redefinir um método
Quase tudo e objeto



\section{Estruturas de controle}
\subsection{if}
 
Exemplo Completo

\begin{lstlisting}[language=ruby]
if count > 10
  puts "Try again"
elsif tries == 3
  puts "You lose"
  puts  Number:"
end
\end{lstlisting}

Exemplo Simples

\begin{lstlisting}[language=ruby]
if radiation > 3000
  puts "Danger"
end
\end{lstlisting}

Modificador de Sentença

\begin{lstlisting}[language=ruby]
puts "Danger, Will Robinson" if radiation > 3000
\end{lstlisting}

\subsection{case} 
\begin{lstlisting}[language=ruby]
print "Enter your grade: "
grade = gets.chomp
case grade
when "A"
  puts 'Well done!'
when "B"
  puts 'Try harder!'
when "C", "D"
  puts 'You need help!!!'
  puts "You just making it up!"
 end
\end{lstlisting}


\subsection{while}
\begin{lstlisting}[language=ruby]
while weight < 100 and numPallets <= 30
  pallet = nextPallet()
  weight += pallet.weight
  numPallets += 1
end
\end{lstlisting}
 
Modificador de Sentença


\begin{lstlisting}[language=ruby]
square = square*square  while square < 1000
\end{lstlisting}

\subsection{for}
\begin{lstlisting}[language=ruby]
for i in 0..5
   puts "Value is #{i}"
end
\end{lstlisting}

\subsection{until} 
\begin{lstlisting}[language=ruby]
until weight >= 100 ||numPallets > 30
  pallet = nextPallet()
  weight += pallet.weight
  numPallets += 1
end
\end{lstlisting}

Modificador de Sentença

\begin{lstlisting}[language=ruby]
square = square*square  until square >= 1000
\end{lstlisting}

\section{Desafio - FizzBuzz}  
Escreva um programa que imprima o número de 1 a 100.
Mas, para múltiplos de três, imprima  ``Fizz'' no lugar do
número e para múltiplos de cinco imprima ``Buzz''. Para
números que são múltiplos de ambos três e cinco
imprima ``FizzBuzz''.
 
\url{http://www.rubyquiz.com/quiz126.html}

\section{Solução FizzBuzz}

\lstinputlisting[language=ruby]{../Exemplos/1-very_basic/fizzbuzz.rb}

\chapter{Containers}

Até agora, os conceitos apresentados são bem comuns em outras linguagens de comunicação. Containers também são
conceitos bastante comuns mas tomemos a linguagem Java. Não existe em Java literais específicos para 
listas. Embora a linhagem tenha array(vetores), uma lista mesma é um objeto sem nenhuma particularidade e
deve ser manipulado como tal.

Para tornar essa discussão mais pálpavel, qual seria a forma mais simples de criar uma lista com os os números de 
1 a 5? A forma mais elementar seria:
\begin{lstlisting}[language=Java]
List<Integer> list = new ArrayList();
list.add(1);
list.add(2);
list.add(3);
list.add(4);
list.add(5)
\end{lstlisting}

Mas, podemos fazer isso de uma forma mais compacta:
\begin{lstlisting}[language=Java]
List<Integer> list = Arrays.asList(new Integer[]{1,2,3,4,5});
\end{lstlisting}
Isso só ocorre porque a linguagem dispõe de uma forma ``confortável'' de inicializar vetores.

Em Ruby, vetores e hash possuem literais próprios e açúcares sintáticos que facilitam sua manipulação. Além disso,
os container fornecem uma alternativa elegante às estruturas de laço (como em Smalltalk) que é mais compatível com o 
paradigma orientado a objetos. Isso faz com que essas classes mereçam um capítulo a parte.

\section{Array}
\begin{lstlisting}[language=ruby]
a = [ 3.14159, "pie", 99 ]
a.type   #        Array
a.length #        3
a[0]     #        3.14159
a << 1
a[3]     #        1
a[-2]    #        99
b = Array.new
b << a   #   [[3.14159, "pie", 99, 1]]
b[0..3] = a    #   [3.14159, "pie", 99, 1]
b[0, 2] = 1    #   [1, 1]
c = %w{a b c d e }  #  =>  ["a", "b", "c", "d"]
\end{lstlisting}


\section{Hash}
\begin{lstlisting}[language=ruby]
h = {'dog' => 'canine', 'cat' => 'feline', 'donkey' => 'asinine'}
h.length        #        3
h['dog']        #        "canine"
h['cow'] = 'bovine'
h[12]    = 'dodecine'
h['cat'] = 99
h        # => {"cow"=>"bovine", "cat"=>99, 12=>"dodecine",
"donkey"=>"asinine", "dog"=>"canine"}
  
a = [[1, 'a'],[2, 'b'],[3, 'c'], [4, 'd']]
b = Hash[a]
# => {1=>"a", 2=>"b", 3=>"c", 4=>"d"}
\end{lstlisting}


\section{Blocos e Iteradores}

Passando blocos
\begin{lstlisting}[language=ruby]
(1..12).each { |i| puts i}
[1, 2, 4].each do |i|
    puts i
end
\end{lstlisting}

Blocos de código
\begin{lstlisting}[language=ruby]
(1..20).each{|x| puts x}
\end{lstlisting}

Influência do Smalltalk:
\begin{lstlisting}[language=smalltalk]
   1 to: 20 do: [:x | x printN1]
\end{lstlisting}

\section{ Métodos de um Enumerable}
\begin{lstlisting}  
all?, any?, collect, detect, each_cons, each_slice, each_with_index, entries,
enum_cons, enum_slice, enum_with_index, find, find_all, grep, include?, inject,
map, max, member?, min, partition, reject, select, sort, sort_by, to_a,
to_set, zip
\end{lstlisting}
 
\section{Exemplos com Enumeraveis}
\begin{lstlisting}[language=ruby]
names = %w{ Frye Leela Zoidberg }
names.find {|name|  name.length>4}          # => "Leela"
names.find_all { |name| name.length > 4}
     #=> ["Leela", "Zoidberg"]
names.grep /oidberg/
# => ["Zoidberg"]
names.group_by {|name|  name.length}
   # =>  {4=>["Frye"], 5=>["Leela"], 8=>["Zoidberg"]}
\end{lstlisting}

\section{Mais exemplos com Enumeraveis}
\begin{lstlisting}[language=ruby]
names = %w{ Frye Leela Zoidberg }
names.map {|name| name.downcase}
# => ["frye", "leela", "zoidberg"]
names.reduce {|acc, name| name.length <= 5 ? acc + name : acc }
# => "FryeLeela"
names.join ", "
# => "Frye, Leela, Zoidberg"
\end{lstlisting}


\chapter{Blocos}

\section{yield}
\begin{lstlisting}[language=ruby]
def proxy_method
  puts "Calling command at: #{Time.new}"
  yield
proxy_method { puts "hello world proxified!"}
#ou com paremtros
def proxy_method
   yield(Time.new)
proxy_method {|time| puts "hello world proxified  at #{time}"}
\end{lstlisting}

\section{call}
\begin{lstlisting}[language=ruby]
def proxy_method(&method)
  # argumento com & precisa ser o ultimo da lista
  puts "Calling command at: #{Time.new}"
  method.call
proxy_method { puts "hello world proxified! "}
#ou com paremtros
def proxy_method (&method)
   method.call(Time.new)
proxy_method {|time| puts "hello world proxified  at #{time}"}
\end{lstlisting}

\section{Proc x Lambda}
\begin{lstlisting}[language=ruby]
fx = Proc.new {|x| x**2}
fxy = proc {|x,y| x+y}
# calling
fx.call(2) # => 4
fxy[2,3,4] #=> 5
fx = lambda {|x| x**2}
fxy = lambda {|x,y| x+y}
# calling
fx.call(2) # => 4
fxy.call(2,3,4) #=> exception na cara!
Proc.new e proc sao equivalentes
\end{lstlisting}


\section{Lambda ``Calculus''}

\begin{lstlisting}[language=ruby, caption="Derivada em Ruby"]
def d(f)
   lambda {|a|
     h = 0.0000000001 # um valor pequeno para h
     h = h * a       if a < 1 && 0 < a
     (f[a+h]-f[a])/h
   }
f = lambda {|x| x**2}
puts d(f)[4]
\end{lstlisting}

\chapter{Objetos em Ruby}
\begin{lstlisting}[language=ruby]
class BookInStock
  def initialize(isbn, price)
    @isbn = isbn
    @price = Float(price)
  end
 
  def to_s
    "ISBN:#{@isbn}, price: #{@price}"
  end
end
stock = BookInStock.new
# ou
stock = BookInStock.new (1234, 10.39)
#invocando metodo
puts stock.to_s
\end{lstlisting}


\section{Variaveis e Escopo}

\begin{tabular}{ l | c | r }
Variáveis Locais & \verb|x name thx1138 _x _26| \\
Variáveis de Instancia & \verb|@name @X  @_ @plan9| \\
Variáveis de Classe & \verb|@@total @@N @@x_pos| \\
Variáveis Globais & \verb|$debug $CUSTOM $_ $plan9| \\
Nomes de Classe & \verb|String BigDecimal| \\
Constants & \verb|FEET_PER_MILE DEBUG|
\end{tabular}

\section{Atributos de instância - forma tradicional}
\begin{lstlisting}[language=ruby]
class BookInStock  
  def isbn
    @isbn
  end
 
  def isbn=(value)
    @isbn = value
  end
 
  def price
    @price
  end
\end{lstlisting}

\section{Atributos de instância - forma declarativa}
\begin{lstlisting}[language=ruby]
class BookInStock  
  attr_accessor :isbn
  attr_reader :price
 end
\end{lstlisting}

\section{Herança}
\subsection{Exemplo de Heranca}
\begin{lstlisting}[language=ruby]
class SpecialStock < BookInStock
\end{lstlisting}

\section{Herança - Singleton Pattern}

\subsection{forma tradicional}
\begin{lstlisting}[language=ruby]
class Logger
  private_class_method :new
  @@logger = nil
  def Logger.create
    @@logger = new unless @@logger
    @@logger
  end
end
\end{lstlisting}

\subsection{módulo Singleton}
\begin{lstlisting}[language=ruby]
require 'singleton'
class Logger
  include Singleton
  
  def initialize
    @log = File.open("log.txt", "a")
  end
  def log(msg)
    @log.puts(msg)
  end
Logger.instance.log('message 2')
\end{lstlisting}

\begin{lstlisting}[language=ruby]
stock =  BookInStock.new
class << stock
   def alter_price
         price * 1.4
   end
\end{lstlisting}

\section{Criando um Enumerable}
  *  Basta implementar o metodo each. 
\begin{lstlisting}[language=ruby]
class Node
  include Enumerable 
  attr_accessor :next, :previous, :v
 def initialize(v = {})
    @v = v
  end
 def to_s
    v.to_s
  end
\end{lstlisting}

\begin{lstlisting}[language=ruby]
linked_list.rb (continuacao)
 def <<(node)
    node.next = self.next
    node.previous = self
    self.next.previous = node unless self.next.nil?
    self.next = node
  end
 def remove
    node = self.previous
    node.next = self.next
    self.next.previous = node
    self
 end
\end{lstlisting}

\begin{lstlisting}[language=ruby]
  def each
    node = self.next
    until node == self || node.nil?
      yield node
      node = node.next
    end
  end

\end{lstlisting}



\chapter{Mais sobre métodos}

\section{Lista de parâmetros}

\begin{lstlisting}[language=ruby]
def my_new_method               # No arguments
  # Code for the method would go here
end

def my_other_new_method(arg1, arg2, arg3)  # 3 arguments
  # Code for the method would go here
end

def cool_dude(arg1="Miles", arg2="Coltrane", arg3="Roach")  # defaults
  "#{arg1}, #{arg2}, #{arg3}."
end
\end{lstlisting}

\section{Truques com parâmetros}

Aridade não definida

\begin{lstlisting}[language=ruby]
def varargs(arg1, *rest)
  "Got #{arg1} and #{rest.join(', ')}"

varargs("one")  # "Got one and "
varargs("one", "two") # "Got one and two"
varargs "one", "two", "three" # "Got one and two, three"

def varargs(arg1, hash)
 puts "#{arg1} - #{hash}"
end

varargs (1, :a => 1)
end
\end{lstlisting}

\section{Array para argumentos}

Expandindo array para parâmetros

\begin{lstlisting}[language=ruby]
def five(a, b, c, d, e)
  "I was passed #{a} #{b} #{c} #{d} #{e}"
end

five(1, 2, 3, 4, 5 )         #  "I was passed 1 2 3 4 5"
five(1, 2, 3, *['a', 'b'])  # "I was passed 1 2 3 a b"
five(*(10..14).to_a)         #  "I was passed 10 11 12 13 14"
\end{lstlisting}

\section{Proc para bloco}

Convertendo proc para bloco

\begin{lstlisting}[language=ruby]
print "(t)imes or (p)lus: "
times = gets.chomp
print "number: "
number = gets.to_i

if times =~ /^t/
  calc = proc { |n| n*number }
else
  calc = proc { |n| n+number }
end

puts((1..10).collect(&calc).join(", "))
\end{lstlisting}

\chapter{Exceptions, Catch and Throw}

\begin{lstlisting}[language=ruby]
opFile = File.open(opName, "w")

while data = socket.read(512)
  opFile.write(data)
end
\end{lstlisting}

\section{Exceptions}

\begin{lstlisting}[language=ruby]
opFile = File.open(opName, "w")
begin
  # Exceptions raised by this code will
  # be caught by the following rescue clause
  while data = socket.read(512)
    opFile.write(data)
  end

rescue SystemCallError
  $stderr.print "IO failed: " + $!
  opFile.close
  File.delete(opName)
  raise
end
\end{lstlisting}

\section{Catching exception}

Nomeando a exceção

\begin{lstlisting}[language=ruby]

begin
  eval string
rescue SyntaxError, NameError => boom
  # OLHA! sem usar o $!
  print "String doesn't compile: " + boom
rescue StandardError => bang
  print "Error running script: " + bang
end
\end{lstlisting}

\section{Ensure}

Garante que um bloco é chamado

\begin{lstlisting}[language=ruby]

f = File.open("testfile")
begin
  # .. process
rescue
  # .. handle error
ensure
  f.close unless f.nil?
end
\end{lstlisting}

\section{Rescuing a Method}

Begin Rescue

\begin{lstlisting}[language=ruby]

def some_method
  begin
    danger_danger
    true # good return
  rescue Error
    false # error return
  end
end
\end{lstlisting}

Better code

\begin{lstlisting}[language=ruby]

def some_method
  danger_danger
  true # good response
rescue Error
  false # error response
end
\end{lstlisting}

\section{Raise Exceptions}

Formas típicas de se lançar uma exceção

\begin{lstlisting}[language=ruby]

 raise # sem mensagem

 # adicionando uma string...
 raise "Missing name" if name.nil?

 if i >= myNames.size
   raise IndexError, "#{i} >= size (#{myNames.size})"
 end

 # passando o stackTrace via Kernel::caller
 raise ArgumentError, "Name too big", caller
\end{lstlisting}

\section{Especializando Exceções}

Declaração

\begin{lstlisting}[language=ruby]

class RetryException < RuntimeError
  attr :okToRetry

  def initialize(okToRetry)
    @okToRetry = okToRetry
  end
end
\end{lstlisting}

Como lançar

\begin{lstlisting}[language=ruby]

def readData(socket)
  data = socket.read(512)
  if data.nil?
    raise RetryException.new(true), "transient read error"
  end
  # .. normal processing
end
\end{lstlisting}

\section{Especializando Exceções II}

Tratanto a exceção

\begin{lstlisting}[language=ruby]

begin
  stuff = readData(socket)
  # .. process stuff
rescue RetryException => detail
  retry if detail.okToRetry
  raise
end
\end{lstlisting}

\section{Catch e Throw}

Desvio incondicional com labels

\begin{lstlisting}[language=ruby]

def promptAndGet(prompt)
  print prompt
  res = readline.chomp
  throw :quitRequested if res == "!"
  return res
end

catch :quitRequested do
  name = promptAndGet("Name: ")
  age  = promptAndGet("Age:  ")
  sex  = promptAndGet("Sex:  ")
  # ..
  # process information
end
\end{lstlisting}

\chapter{Módulos}
Uso

\begin{enumerate}
  \item Criar namespace (evitar conflito de nomes)
  \item Mixin (permitir herança de traços – como se fosse uma cópia do conteúdo do módulo no local incluído)
\end{enumerate}

\section{Declaração}

\begin{lstlisting}[language=ruby]

module Trig
  PI = 3.141592654
  def Trig.sin(x)
   # ..
  end
  def Trig.cos(x)
   # ..
  end
end
\end{lstlisting}

\subsection{Uso}

\begin{lstlisting}[language=ruby]

require "./trig"
puts Trig.sin(Trig::PI / 3.0)
\end{lstlisting}

\section{Mixins}

Applying mixin

\begin{lstlisting}[language=ruby]

class BigInteger < Number
  # Adiciona metodos de instancia de Stringify
  include Stringify

  # Adiciona metodos de classe de Math
  extend Math

  # Adiciona um constructor com um parametro
  def initialize(value)
    @value = value
  end
end
\end{lstlisting}

\section{Applying mixin}

\begin{lstlisting}[language=ruby]

bigint1 = BigInteger.new(10)

puts bigint1.intValue   # --> 10

bigint2 = BigInteger.add(-2, 4)
puts bigint2.intValue   # --> 2

puts bigint2.stringify   # --> 'Two'

bigint2.extend CurrencyFormatter
\end{lstlisting}

\chapter{Pacotes Básicos}

\section{BigDecimal}

\begin{lstlisting}[language=ruby]

require 'bigdecimal'

BigDecimal.new('1.23) # => #<BigDecimal:7ffe0b052bc8,'0.123E1',18(18)>
\end{lstlisting}

\section{OpenStruct}

\begin{lstlisting}[language=ruby]
require 'ostruct'
\end{lstlisting}

\section{Test}

\begin{lstlisting}[language=ruby]

require "test/unit"

class TesteFoo  < Test::Unit::TestCase

  def test_eFoo_foo
    assert_same(1, 0, "Que pena")
  end
end
\end{lstlisting}

\section{ERB}

\begin{itemize}
  \item Sistema de Template padrão do Ruby
  \item Uma classe como outra qualquer
  \item Via linha de comando é possível parsear um arquivo erb
\end{itemize}

\begin{lstlisting}[language=ruby]
require 'erb'

template = ERB.new('1 + 1 = <%= 1 + 1 %>')
template.result # => '1 + 1 = 2'
\end{lstlisting}

\section{Net::HTTP}

\begin{lstlisting}[language=ruby]
	require "net/http"
	require "uri"
	require 'methodize'
	
	def get_page (string)
	  uri = URI.parse(string)
	  response = Net::HTTP.get_response(uri)
	  response.body
	end
\end{lstlisting}

\section{JSON}

\begin{lstlisting}[language=ruby]
	require "net/http"
	require "uri"
	require 'json'
	require 'methodize'
	
	def get_page (string)
	  uri = URI.parse(string)
	  response = Net::HTTP.get_response(uri)
	  json = JSON.parse(response.body)
	  json.extend(Methodize)
	end
\end{lstlisting}

\section{YAML}

\subsection{Arquivo yaml}

\begin{verbatim}
	simple symbol: !ruby/symbol Simple 
	shortcut syntax: !ruby/sym Simple 
	symbols in seqs: 
	  - !ruby/symbol ValOne 
	  - !ruby/symbol ValTwo 
	  - !ruby/symbol ValThree 
	symbols in maps: 
	  - !ruby/symbol MapKey: !ruby/symbol MapValue 
\end{verbatim}

\subsection{Ruby code}

\begin{lstlisting}[language=ruby]
	require "yaml"
	
	config = YAML.load_file("config.yml") # From file
	p config
\end{lstlisting}

\subsection{Result}

\begin{verbatim}
{"simple symbol"=>:Simple, "shortcut syntax"=>:Simple, 
"symbols in seqs"=>[:ValOne, :ValTwo, :ValThree], 
"symbols in maps"=>[{:MapKey=>:MapValue}]}
\end{verbatim}


\bibliography{bibliografia}
\bibliographystyle{plain}

\end{document}
